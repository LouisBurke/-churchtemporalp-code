\documentclass[a4paper,10pt]{article}
\usepackage{algorithm}
\usepackage{algorithmic}
\usepackage{url}
\usepackage{listings}
\usepackage{graphicx}

\usepackage[left=2.54cm,top=2.54 cm,right=2.54cm,nohead,nofoot]{geometry}

\begin{document}

\title{\bf Traffic Light Phase Optimisation using GA}
\author{{\bf Louis Burke}\\ \\
School of Computer Science \& Informatics\\
University College Dublin\\
louis.burke@ucdconnect.ie}
\maketitle

\begin{abstract}
Road Traffic congestion is a major problem for road networks especially for those with traffic light controlled junctions. This paper 
explores the use of an evolutionary strategy for determining the optimal traffic signal phase times in a sequence, for an irregular 
shaped road network (i.e.\ non-grid plan) in order to reduce congestion. Fitness is determined to be the mean time it takes for all cars 
to finish their routes (random) through the network. The candidate solutions are generated using a Genetic Algorithm (GA). To evaluate 
the fitness of each member solution in the population an open source traffic simulator SUMO \cite{SUMO} is used. It was found that a
GA could come up with good solutions for ensuring throughput of vehicles in the network and would be suitable for providing inspiration
for setting up traffic light signals for a given road network.

\end{abstract}

\section{Introduction}
Is a GA strategy useful in reducing congestion in an irregular road network by evolving the traffic light signal phases?
For a given road network comprised of 6 junctions with traffic lights and a random set of vehicles routed through 
that network, can a GA, operating on all the traffic lights as a chromosome, improve the time taken for the vehicles
 to get through the network. This question assumes a traffic light control system for each junction which is allocated
 arbitrarily and independently of each other traffic light, thereby leaving room for optimisation using a GA.

Congestion has negative impact on utilization of the transportation infrastructure, increasing travel time, air pollution and fuel 
consumption some of the root causes of this congestion lie in the traffic management of the vehicles going through the network namely
the traffic lights at junctions. This problem could be seen as being related to queuing and it would be interesting to see if mean time 
for vehicles waiting or queuing at junctions is reduced. 
This project question takes inspiration from Multi-agent system based urban traffic management \cite{Multi-agent}. In that paper they 
described the use of a GA to evolve green signal phases times in an online manner to improve traffic flow throughout the road network. 
That is to say, given the current or prevalent flow of traffic, evolve a traffic light solution to improve the network flow. This was
done over the green signal times. The ability to do this is outside the scope of this project to achieve. I instead decided to encode 
the entirety of a simulated road network’s traffic lights into a chromosome and evolved it using steady state GA provided by the C++ 
GA library  \cite{galib}.  By doing this I wanted to show the power of a GA to search for a good solution for traffic light signals to 
improve traffic throughput. 

This was then tested when cars with random routes tried to get through the network. GA is a search heuristic that mimics the process of
natural evolution. This heuristic is routinely used to generate useful solutions to optimization and search problems. GA uses an 
iterative process that incorporates variation, selection and inheritance. The algorithm creates a `population' of possible solutions, 
selecting the best individuals to create the next generation. The process is driven by a fitness function, which is used to evaluate the 
solutions.  In this case the mean time taken for all vehicles to complete their route symbolises a solution's fitness. Thus a fitness 
function only specifies how to evaluate the output, not how to accomplish it. By allowing GA's to explore the problem space, they are 
capable of creating novel solutions. For more information on GA’s see \cite{ANASbook} \cite{GAbook} \cite{IntroGA} \cite{evocomp}.

GA is an appropriate heuristic or algorithm for this kind of problem because of the discrete nature the traffic light signal times and the traffic light phase description. That is to say that the traffic light colour and phase duration time is easily encoded in binary as detailed later on in this paper.

Genetic Algorithms exhibit high performance for finding near-optimal solutions rapidly with respect to large and dynamically changing search spaces. Their inherent strength as optimization techniques make them good candidates for solutions to road traffic management and congestion avoidance problems \cite{dynam} \cite{genren}.

\section{Genetic Algoritms}
Genetic algorithms use the essence of the mechanisms of heredity and evolution, extracting these processes from the specific context of genetics. By using these concepts, it is possible to design adaptive systems. These algorithms search any suitably represented domain of structures' for
ones with higher associated measures of performance. They do so by maintaining a population of structures together with their associated performance measures and repeatedly applying idealized genetic operators to structures in the population to produce new structures for testing. The structures to be operated upon are chosen from the population based on their associated performance measure; therefore, structures exhibiting better performance have a higher probability of producing new structures. The genetic operators are such that characteristics of a structure associated with good performance are preserved and propagated through the population over time.

\subsection{GA Pseudo Code}
Pseudo code for a G.A.s is as follows.
\newline
\newline
\textbf{begin}
\newline
Randomly generate an initial population with M structures.
\newline
\begin{tabbing}
\textbf{For}  \= each M in population, evaluate and save its performance measure.\\
\> \textbf{while} \= performance criterion is not met \\
\> \> \textbf{do begin} \= \\
\> \> \> Generate the next population by selecting structures from current\\ 
\> \> \> populations with selection probability and applying genetic operators\\ 
\> \> \> to them (eg.\ the structures that have higher performance will have\\ 
\> \> \> a higher probability of reproducing).\\
\> \> \textbf{For} each M in the new population, \\
\> \> \> Evaluate performance and save their performance measures.\\
\> \> \textbf{endbegin;}\\
\> \textbf{endwhile;}\\
\textbf{endbegin;}
\end{tabbing}

\subsection{Genetic Operators}
The genetic operators are the means of introducing new rules to the system. The following is an introduction to two primitive genetic 
operators.
\newline
\newline
\newline
\newline
\subsubsection{Crossover}

The crossover operator recombines two knowledge structures from the population by exchanging string segments. Therefore, the crossover 
operator introduces the “in-betweencharacteristics” of the selected two structures. In terms of search strategy, crossover operators 
have the ability to concentrate the search upon interesting areas. A crossover point in a selected structure is selected at random. 
Of course, the structures that have higher performance measures will have a higher probability of participating in a crossover.
\newline
\newline
For example:
\newline
\newline
Structure 1: a b e f g h i
\newline 
Structure 2: L K U T A J X
\newline
\newline
The Crossover of points 3 and 5 will be
\newline
\newline
Result: a b \textbf{U T A} h i and L K \textbf{e f g} J X

\subsubsection{Mutations}
In producing new strings, the crossover operator draws only on the information present in the population. In other words, the crossover
operator cannot generate entirely new genotypes; it may only deal with genotypes which already exist. The mutation operator provides a
means for introducing missing information into the population. It generates a new structure by modifying the values of one or more 
positions (gene) in an existing structure. For example:
\newline
\newline
Structure: 1 0 0 1 1 1 0
\newline
\newline
The mutation of the 2nd and 5th points would be one of the following.
\newline
\newline
Result:\newline
1 \emph{1} 0 1 \emph{1} 1 0\newline
1 \emph{1} 0 1 \emph{0} 1 0\newline
1 \emph{0} 0 1 \emph{1} 1 0\newline
1 \emph{0} 0 1 \emph{0} 1 0
\newline
\newline
The above is taken from:
\url{http://ieeexplore.ieee.org/stamp/stamp.jsp?tp=&arnumber=40288}

\subsection{GA and the traffic problem}
In the \emph{Multi-agent system based urban traffic management} \cite{Multi-agent} paper it described the use of various algorithms that
have been applied to traffic management problems. These include for example Learning Classifier Systems, Cellular Automata etc.\ all of
which work with evolutionary algorithms. This further illustrates the power of GA’s when applied to problems related to traffic 
management. The work done in \cite{Multi-agent} seeks to provide an extension to the approach taken in the paper \cite{realcode},
developed for queue reduction in a single intersection, to a larger network having six intersections.

Whereas my model focused on the throughput of vehicles through the road system, the model used in \cite{Multi-agent} focused on the 
dissipation and formation of retained vehicles in the different lanes at the end of different phases.
\newline
\section{SUMO Simulation of Urban MObility}
``Simulation of Urban MObility'' (SUMO) \cite{Sim4} is an open source, highly portable, microscopic road traffic simulation package 
designed to handle large road networks. In traffic research, four classes of traffic flow models are distinguished according to the 
level of detail of the simulation, macroscopic, microscopic, mesoscopic and submicroscopic. Macroscopic simulations concern the modeling
of traffic flow, microscopic models simulate the movement of every single vehicle on a given street, mesoscopic resides between 
microscopic and macroscopic simulations, where vehicle movement is mostly simulated using queue approaches and single vehicles are moved
between such queues. Sub-microscopic models regard single vehicles like microscopic, but extend them by dividing them into further 
substructures, which describe the engine's rotation speed in relation to the vehicle's speed or the driver's preferred gear switching 
actions, for instance. SUMO is a microscopic, multi-modal traffic simulation. It allows one to simulate how a given traffic demand which
consists of single vehicles moves through a given road network. The simulation allows one to address a large set of traffic management 
topics. It is purely microscopic: each vehicle is modelled explicitly, has its own route, and moves individually through the network. 
A description of a route is given below in XML with an accompanying legend.


\begin{lstlisting}
<vehicle id="t0" depart="0.00">
    <route edges="9wo 3fi 2o 9o"/>
</vehicle>
\end{lstlisting}

\begin{center}
    \begin{tabular}{| l | l | p{10cm} |}
    \hline
    id & string & The name of the route. \\ \hline
    edges & list & The edges the vehicle shall drive along, given as their ids, separated using spaces. \\ \hline
    \end{tabular}
\end{center}

Also a description of a SUMO edge (typical for all edges in the network used for this project) is given below.

\begin{lstlisting}
<edge id="0si" 
      from="13" 
      to="3" 
      priority="1" 
      numLanes="3" 
      speed="50.00"/>
\end{lstlisting}

\begin{center}
    \begin{tabular}{| l | l | p{10cm} |}
    \hline
    id & string & The name of the edge. \\ \hline
    from & referenced node id & The name of a node within the nodes-file the edge shall start at. \\ \hline
    to & referenced node id & The name of a node within the nodes-file the edge shall end at.\\ \hline
    priority & int & The priority of the edge. \\ \hline
    numLanes & int & The number of lanes of the edge; must be an integer value. \\ \hline
    speed & float & The maximum speed allowed on the edge in m/s; must be a floating point number. \\ \hline
    \end{tabular}
\end{center}
 
All edges connect nodes. A node may be either a traffic light junction or a terminal node. Put altogether the network formed,
for this study, in SUMO will look like the following image Figure 1.

\begin{figure}[h!]
  \caption{Visualisation of the SUMO Network.}
  \centering
    \includegraphics[scale=0.5]{sumo_network}
\end{figure}

Almost every facet of SUMO is driven by XML, making it very easy to interface with other applications and simple to change between 
executions of the simulator. All time values are given in seconds. All length values are given in meters. This means that speed is 
given in m/s. Each simulation time step is one second long, in other words; in each simulation step, one second real time is simulated.

\subsection{Building the network}
As mentioned before the entire of the network I’m simulating using SUMO is defined in a collection of XML files. The three files are:
\newline
\newline
rou.xml\newline 
net.xml\newline
sumo.cfg\newline
\newline
sumo.cfg points to both net.xml and rou.xml.\ net.xml contains the network definition as seen above and rou.xml contains the random routes
of vehicles for the network.

In order to make the binary string symmetrical i.e.\ to make every junction encode-able conforming to an 8 by 66 array of bits, some of 
the junctions had to be moved from the original intended network. This is down to the fact that depending on the shape of the junction 
the number of lights needed for a given phase was different therefore making the length the genome non uniform. An example of a junction 
from the SUMO network used in this study is shown in Figure 2 below.
\newline
\begin{figure}[h!]
  \caption{Visualisation of the SUMO Network Junction.}
  \centering
    \includegraphics[scale=0.333]{junction}
\end{figure}
\newline
\subsection{Encoding the chromosome for a steady state GA}
The SUMO simulator, in order to construct the network to run the simulation, uses a set of specifications encapsulated in an XML file 
called a net file. In other words the way in which the network looks and works is stored in a large XML file. The part of the file of 
particular interest is the area concerned with traffic light control.  A section of the XML used in this study looks like this.
\newline
\newline
\newline
\newline
\newline
\newline
\begin{lstlisting}
<tlLogic id="0" type="static" programID="0" offset="0">
    <phase duration="2" state="rrryrrrgGrGrGrrrrrrr"/>
    <phase duration="4" state="yrgGrrgrygGGygrrGrGg"/>
    <phase duration="4" state="rrGrGrrrrrggGrygGGyg"/>
    <phase duration="1" state="rrrrGyGgrgrrrrggrrrG"/>
    <phase duration="8" state="rrrryyrrGGGgrrGryrrr"/>
    <phase duration="7" state="ygGggrrGrrrGrrryrryy"/>
    <phase duration="4" state="yrrrGrgrgGrryryrryry"/>
    <phase duration="5" state="rGgryrrrygrrrryggyry"/>
</tlLogic>
\end{lstlisting}

\begin{center}
    \begin{tabular}{| l | l | p{10cm} |}
    \hline
    id@tlLogic & string & The id of the traffic light. \\ \hline
    type & enum & The type of the traffic light. \\ \hline
    programID & string & The id of the traffic light program. \\ \hline
    offset & int & The initial time offset of the program \\ \hline
    \end{tabular}
\end{center}

\begin{center}
    \begin{tabular}{| l | l | p{8cm} |}
    \hline
    duration@phase & time (int) & The duration of the phase. \\ \hline
    state@phase & list of signal states id & The traffic light states for this phase. \\ \hline
    \end{tabular}
\end{center}

The XML above describes the traffic light sequence, in this case, for one of the six junction’s traffic lights, namely traffic light 
junction id 0. The phase duration indicates the length of time a particular phase lasts for, the state describes what the light 
configurations look like for that duration. To encapsulate this in a genome it was decided that a 2D binary string be used. The maximum 
phase duration was limited to 60 seconds for this experiment thus 6 bits were used to encode it, if a particular genome had a value 
contained within those 6 bits that was greater than 60 then 60 was taken as default. Next each traffic light colour was given 3 bits and 
mapped as follows r = 000b, y = 001b, g = 010b, G = 011b. Concretely red, green, blue, Green are 0,1,2,3 respectively. The default value 
for a light if it fell out of this range i.e.\ greater than 3 was 000b. There are 20 lights per junction so that requires 60 bits in order
to store the state. To sum up the length of the genome is 66 bits, an example of the bit string shown here looks like the following:
\newline
\newline
msb
1000011001000110111011001111001101100110111011100101100010110101
                                                             lsb 
\newline
\newline
Considering there are 8 phases per traffic light junction and there were/are 6 junctions means the traffic light chromosome for the 
network is a 66x48 2D binary array.

The number of vehicles for every execution of the simulator was 250.


\section{GAlib}
Instead of writing a GA implementation myself I decided to use an open source C++ GA library called GAlib. GAlib is a set of C++ genetic 
algorithm objects. The library includes tools for using genetic algorithms to do optimization in any C++ program. Essentially I wrote a 
C++ program that setup a steady state GA, then for every individual in a population wrote it out as XML to the net.xml file then ran 
SUMO to ascertain fitness values for each individual. This was very straight forward using C++ string and bitset functionally.

\subsection{Steady State GA}
The generational genetic algorithm creates new offspring from the members of an old population using the genetic operators and places 
these individuals in a new population which becomes the old population when the whole new population is created. The steady
state genetic algorithm is different to the generational model in that there is typically only a select number of new individuals are
inserted into the new population at any one time.

\subsection{GA setup}
The selection method for the GA, was Roulette wheel. The GA direction was minimise i.e.\ the fitness function returned a value to the GA 
which the GA was always trying to minimize. This float value was parsed from the output file generated by SUMO on the command line.

\subsection{GA Parameters}
Following are the set of parameters used in the \@GA. The number of generations was 200. The population size was 100. The replacement 
every generation was 50 out of every 100, 0.5 being the probability of replacement. The mutation rate was 0.01. The probability of 
crossover was 0.9. The probability of convergence was set to 0.99.

\section{Testing}
60 runs in total were executed and the fittest individual was taken and tested against, the SUMO stock traffic light configurations, 
and also against a random configuration, the result of which can be seen below Figure 3.

\begin{figure}[h!]
  \caption{Fitness after 200 generations averaged across 60 runs.}
  \centering
    \includegraphics[scale=0.4]{errorplot}
\end{figure}

\subsection{Generating a random configuration} 
For testing purposes 10 runs of 1 generation were done and the best out of the population of 100 was returned, then the best out of the
10 runs was chosen as a way of providing a random sample of possible traffic light configurations to test against the evolved solution.
	
The results can be seen below and will be explained in the next section.
\newline
\newline
\newline
\newline
\subsection{Test Results}

As can be seen from Figure 4, the GA evolved traffic light logic(GA Evolved TLL) was consistently faster than the randomly generated,
and SUMO allocated TLL solutions. However although I noticed an improvement in network throughput, it turns out that in some cases 
there was unacceptable wait times for a number of vehicles, roughly speaking 15 out of 250. When this happens the vehicle is removed 
from the simulations. The table A illustrates this problem, essentially there are more cars taken from the simulation due to 
unacceptable wait time from the GA Evloved TLL networks then from the SUMO allocation situations where ever there is a * beside the 
GA Evloved TLL time. When visualising the execution of the fittest TLL network using SUMO GUI functionality it was interesting to 
note that congestion seemed to gather around the top two junctions of the network (see Figure 1).

Another important thing to note was that the evolved solution outperformed the other solutions consistently, 
giving confidence in the idea that it found a good solution in general for the network. Even with the SUMO allocation and the 
generation 0 (or random) allocation of lights there were instances of vehicles waiting too long in the simulation. Worryingly this was
often the case with the SUMO allocation. When visualising this in sumo-gui this is manifested by a car waiting at a junction even though 
it has a chance to continue it's route.
\newline
\begin{figure}[h!]
  \caption{Results.}
  \centering
    \includegraphics[scale=0.7]{test_results}
\end{figure}

\begin{figure}[h!]
  \caption{The * denotes a situation where the number of cars expelled from the GA network definition was greater than the SUMO traffic light logic.}
  \centering
    \includegraphics[scale=0.6]{compare}
\end{figure}

\section{Discussion}
If this was applied on an online fashion using my setup it would be too slow to keep up with the pace of change in the traffic 
conditions, due to the limitations of computation power and the fact that a simulation is being run each time to ascertain fitness. 
A better way to decide fitness would need to be found.
	
The relevance of GA in this instance is that it can take into account inter-junction behaviour as demonstrated to make sure that one 
solution for one set of traffic lights doesn’t degrade the traffic handling capabilities of another.

It would be interesting to see how GA could be used on differing aspects of a network control especially if fitness could be determined 
by the congestion or lack thereof at junctions by means other than those described in this paper. Most of the work done before on 
congestion management with GAs makes use of Learning Classifier Systems etc. 

As mentioned earlier the articles \cite{Multi-agent} and 
\cite{realcode} focused on the dissipation and formation of retained vehicles in the different lanes at the end of different phases. It 
would be interesting to explore this further but in the context of using \@SUMO. That is to say it would be interesting to explore, the
features of SUMO that would allow you to get a measurement of the conditions of inidividual junctions so as to enable the creation of
fitness measures like those explored in \cite{Multi-agent}. It's also a relevant investigation because although the expectation was that
all junctions would perform optimally with the GA, as mentioned before, some junctions incurred long delays relative to others.


\section{Conclusion}
In this paper I set out to explore the use of Genetic Algorithms for determining the optimal traffic signal phase times in a sequence,
for an irregular shaped road network in order to reduce congestion. The results showed that the GA was capable of evolving traffic signals
quite easily. 

The results also showed that GA significantly outperformed random search and SUMO allocated light logic. The implmenetation I used, in this 
study, may have more relevance in a larger network then used for this project. It would be interesting to see use made of the induction
loop features of SUMO so as to explore using GA in an online fashion. That is to say to use the induction loop data and the resultant 
indication of congestion to develop a fitness function similar to that seen in \cite{Multi-agent}.

\bibliographystyle{ieeetr}
\bibliography{myrefs}
\end{document}
